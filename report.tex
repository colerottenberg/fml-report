\documentclass[conference]{IEEEtran}
\IEEEoverridecommandlockouts
% The preceding line is only needed to identify funding in the first footnote. If that is unneeded, please comment it out.
\usepackage{cite}
\usepackage{amsmath,amssymb,amsfonts}
\usepackage{algorithmic}
\usepackage{graphicx}
\usepackage{textcomp}
\usepackage{xcolor}
\def\BibTeX{{\rm B\kern-.05em{\sc i\kern-.025em b}\kern-.08em
    T\kern-.1667em\lower.7ex\hbox{E}\kern-.125emX}}
\begin{document}

\title{Final Report\\}

\author{\IEEEauthorblockN{1\textsuperscript{st} Cole Rottenberg}
\IEEEauthorblockA{\textit{Electrical and Computer Engineering} \\
\textit{University of Florida}\\
Gainesville, USA \\
colerottenberg@ufl.edu}}

\maketitle

\begin{abstract}
A summary description of the contents of the report and your findings.
\end{abstract}

\section{Introduction}
%Overview of your experiment/s and a literature review. For the literature review, include any references to any relevant papers for your experiment/s. So, whatever you decide to do, search the ACM and IEEE (or other) literature for relevant papers to read and refer to.

Research into the use of fundamental machine learning techniques in order to accurately classify images of American Sign Language (ASL) letters has been a topic of interest for many years. I began working with this limited dataset in order to explore the potential of using machine learning to classify images of ASL letters. The dataset consists of 9 classes, each corresponding to a different letter in the ASL alphabet. The dataset is relatively small, with only 1440 images in total with an 80/20 split for training and testing. The images are 100x100 pixels in size and have 3 RGB channels. The goal of this project is to explore the potential of using machine learning to classify these images and to experiment with different techniques to improve the accuracy of the model. The first approach came with exploring the use of basic convulation neural networks (CNNs) to classify the images. Built onto this, I experimented with handcrafted sobel filters to preprocess the images before feeding them into the CNN. The second direction taken was exploring the generational leap of using deep learning techniques from AlexNet, VGG16, and ResNet50. Research by has shown that these models have been successful in classifying images in the ImageNet dataset. \cite{DBLP:journals/corr/HuangLW16a}

\section{Implementation}
Describe and outline any specific implementation details for your project. A reader should be able to recreate your implementation and experiments from your project report. If you participate in the extra credit contest, be sure to describe the methodology on how you will identify unknown classes that were not in the training data.

\section{Expirements}
Carefully describe your experiments with the training data set and any data augmentation set you constructed or existing data sets. Include a description for the goal of each experiment and experimental findings. This is the bulk of what you will be graded on - if your experimental design is not sound or your experiments do not make sense, you will lose points.

\section{Conclusion}
Describe any conclusions or things you learned from the project. Your conclu- sions must follow from what you did. Do not copy something out of a paper or say something that has no experimental support in the Experiments section.

\section{References}
Listing of all references in IEEE bibliography format.

\section*{References}

\bibliographystyle{IEEEtran}
\bibliography{references}

\end{document}
